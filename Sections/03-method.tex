
%*----------- SLIDE -------------------------------------------------------------
\begin{frame}[fragile, t]{Primeiros passos}
    \begin{enumerate}
        \item Instalar Docker
            \href{https://docs.docker.com/engine/install/ubuntu/}{Instalação para Ubuntu}
        \item Hello world
        \begin{lstlisting}
            $ sudo docker run hello-world
        \end{lstlisting}
        \item Instalar extensão para VSCode
            ID: ms-azuretools.vscode-docker
    \end{enumerate}
%*----------- notes
    \note[item]{Notes can help you to remember important information. Turn on the notes option.}
\end{frame}
%-
%*----------- SLIDE -------------------------------------------------------------
\begin{frame}[fragile, t]{Comandos básicos}

    \begin{lstlisting}
        # Extrair/enviar uma imagem para um repositorio
        $ docker pull ubuntu:bionic
        $ docker push ubuntu:bionic25

        # Cria e inicializa um container gravavel 
        $ docker run [OPTIONS] ubuntu:bionic bash

        # Pausa/Inicia um ou mais containers
        $ docker stop NAME
        $ docker start NAME

        # Executa um comando em um container em execucao
        $ docker exec -ti NAME 

    \end{lstlisting}
%*----------- notes
    \note[item]{Notes can help you to remember important information. Turn on the notes option.}
\end{frame}
%-
%*----------- SLIDE -------------------------------------------------------------
\begin{frame}[fragile, t]{Comandos básicos}

    \begin{lstlisting}
        # Mostra as imagens criadas,tamanho,repositorio,tag 
        $ docker image ls

        # Lista containers
        $ docker ps [OPTIONS]

        # Para fechar um terminal integrado com o container
          "Ctrl+D"

        # Remove um ou mais containers/imagens
        $ docker rm [OPTIONS] NAME 
        $ docker rmi [OPTIONS] NAME_IMAGE 


    \end{lstlisting}
%*----------- notes
    \note[item]{Notes can help you to remember important information. Turn on the notes option.}
\end{frame}
%-
%*----------- SLIDE -------------------------------------------------------------
\begin{frame}[fragile, t]{Docker+ROS}
    1- Iniciar um \textit{container} com a imagem do ros que deseja; 
    
    2- Iniciar um \textit{container} com a imagem do Ubuntu necessário para o ros e então \href{http://wiki.ros.org/kinetic/Installation/Ubuntu}{instalar} o ros normalmente
    \begin{lstlisting}
        docker pull ubuntu:xenial
        docker run --name teste -ti ubuntu:xenial  bash
        apt-get update && apt-get install -y lsb-release wget && apt-get clean all  

        # seguir com os passos da instalação do ros

    \end{lstlisting}
%*----------- notes
    \note[item]{Notes can help you to remember important information. Turn on the notes option.}
\end{frame}
%-
%*----------- SLIDE -------------------------------------------------------------
\begin{frame}[fragile, t]{GUI's+ROS}
    # explicar a forma mais segura
    \begin{lstlisting}
        docker run -it \
            --env="DISPLAY" \
            --env="QT_X11_NO_MITSHM=1" \
            --volume="/tmp/.X11-unix:/tmp/.X11-unix:rw" \
            ubuntu:xenial \
        export containerId=$(docker ps -l -q)

        xhost +local:`docker inspect --format='{{ .Config.Hostname }}' $containerId`
        docker start $containerId

    \end{lstlisting}
%*----------- notes
    \note[item]{Notes can help you to remember important information. Turn on the notes option.}
\end{frame}
%-